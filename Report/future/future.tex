\chapter{Future Work}
\label{chp:6}
\section{System Limitations}
One of the main limitations of the system is the direct correlation between system performance and quality of input corpus. The system relies on the LDA topic model representation to provide term topic relations which are normally used by ontology based summarisation systems to provide domain specific personalised summaries. The proposed system is limited by how well LDA is able to create a topic model that accurately represents the content of the corpus. As discussed in the evaluation this system is limited by the size and content of corpus as both relate directly to LDA ability of producing an accurate topic model. Topic models accuracy can be improved by extraction of concepts, but the method used for concept extraction in this system is limited, by its use of nouns and named entities without centralisation. So in order for the proposed system to achieve its best performance an ideal corpus of domain must be used. An ideal corpus is one that is representative of all the material specific to a domain and nothing more. An ideal corpus would create the best possible topic representation, improving retrieval and subsequently improving the personalisation performance of the summary. Another limitation is the need for LDA to create a model that requires some parameter tuning. One thing that was not evaluated to a high extent is what parameters provided the highest performance of summarisation. The system did perform well without extensive tuning, suggesting that it might not be necessary, but better summarisation performance could be achieved with a better parameterized LDA model.

Another limitation of this system is the approach to personalisation. The method used personnalises a summary at the document level. Document level personalisation forms a summary that is more generalised the one that personnalises at the summary level. More general summaries limit are likely to form less relevant general summaries to highly specific queries. This means that this system is better used in assisting a reading to content they require then replacing the need to examine the content directly. 

\section{Design}
The design presented here is just one of many possible systems that could be constructed from existing methods of summarisation that would perform domain specific personalised summarisation free of formal ontologies. Much more work could be done to explore other combinations of methods to construct similar systems. The design process which identified the classification and tasks of the needed system could be explored with other methods in an attempt to determine a system that performs better than the system proposed by this project. 

The most crucial improvement that could be made for the proposed system is adjustments to the LDA topic model to better serve the content it is modeling.  The proposed system's main limitation is how well a domain corpus could be accurately represented using a LDA topic model. While some of the limitations of the system are inherent to a LDA based topic presentation, the LDA implementation presented in this system is direct. Therefore an tailored implementation of LDA which addresses the limitations of a direct LDA implementation, could offer a great improvement is personalisation of summaries from a more accurate representation of a corpus. Other intermediate representations such as document graph based methods could also be explored, but supervised methods such as machine learning methods should be avoided for failing to be applied to unseen domains.

Another area of the system which could be explored to possibly improve the proposed system, is t directly personalising a summary at the summarisation level. The method used by the proposed system performs personalisation at the document level, while this method allows for use of performance unsupervised summarization methods, directly scoring sentences using the topic representation might provide more specific summarisation which would improve the performance of the system in serving personalised domain specific information needs.

Lastly the system could be implemented with a recommender system that uses the same topic model to model user knowledge as is used for modeling topic in the corpus. A recommender system would have a better interpretation of the intermediate representation and thus might provide better summaries from the uses of more related terms in queries. 

\section{Evaluation}
As mentioned in the literature review the methods of evaluation used for automatic text summarization are limited. Thus more extensive testing could be done to determine the extent to which the proposed system performs personalised domain specific summarisation without formal ontologies. 

One area in which the proposed system could be further tested, is to test the implementation of individual methods. This project implemented both summary formation and components of document retrieval from explanations of methods presented in their respective published papers. Therefore the implementation of these methods may be failing to meet the performance of the methods that were selected due to possible issues with their implementation in this system. By extensively testing the implementation of methods, the performance of the combination of these methods in the system could be better addressed.

Another area of further work in evaluation of the proposed system, would be the testing of the system on multiple corpuses, varying in size and domain comprehension. This would better identify the limitations of the proposed system as well better determine the extent to which the system can perform domain personalised summarisation without formal ontologies.  The system could also be tested on domain specific query-based summarisation datasets to quantifiably address the system’s personalisation performance. 

Lastly, future work could evaluate the system comparatively against ontology based methods in order to directly address the system efficacy in performing domain specific summarisation.
