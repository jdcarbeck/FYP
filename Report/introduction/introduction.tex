\chapter{Introduction}

\section{Motivation}
\justify
Over half of the world has access to the internet, and every one of those internet users faces information overload. Information overload is defined as the difficulty in dealing with an information load of great quantity, complexity, redundancy, contradiction and inconsistency \citep{gross1964managing,roetzel2019information}. Ever since humans have created information, there have been systems developed to handle the storage and retrieval of that information. With modern information and communication technologies the amount of information is exploding and so is the problem of information overload. Approaches to remedy information overload aim to reduce  the amount of incoming information to a recipient as well as enhance the recipient’s information processing capabilities \citep{soucek2010coping}. Most of the content available on the internet is unstructured such as: images, videos, and bodies of text. Thus systems that are developed to help reduce information overload must therefore be able to represent the content of unstructured information in order to determine relevance to limit information and to enhance user processing capabilities.

Information retrieval (IR) systems attempt to retrieve unstructured information from a large collection, based on an expressed information need of a user. Most commonly users express their information needs via queries. IR relies on two tasks that impact the effectiveness of the retrieval systems. Inorder for information to be retrieved the information must be categorised and represented based on the content they contain. The degree of comprehension of the content in the representation affects the system's ability to identify relevant content \citep{chiaramella2000information}. The user must also be able to express their information needs in comprehensible form to the IR system, or the system must be able to extrapolate information needed from natural language processing and query context \citep{carpineto2012survey}. One of the most visible forms of IR systems are search engines like Google. Search engines attempt to perform IR on large volumes of content that exists on the internet, but even well expressed and specific queries produce millions of relevant results, reducing content presented to the users but not significantly mitigating information overload. Thus IR systems have been developed to reduce content further beyond list of relevant material and have been implemented to serve specific domains performing retrieval on smaller sets of information. 

Domain specific IR systems aim to improve the form of the retrieved information as well as relevance of documents to the specific information needs of the user. Within a specific domain this can be done with use of domain specific models that represent topic and terms and their relations. Domain models take the form of ontologies or knowledge bases. Domain models are used in IR systems to expand queries from semantic term relations, extract topics of content, and can be used to form abstract representations of retrieved content in the form of abractive summaries. 

Automatic text summarisation is a form of IR that creates a summary from one or many documents, maintaining key information from the original text(s). Automatic text summarisation is an effective method for reducing information overload as summaries represent relevant information in a digestible condensed form. Text summarisation can also be query based, allowing for users to request a summary based on an information need. Query text summarisation can be further enhanced by models of user's knowledge or interest in a domain. Using user models, personalised summaries are produced with regard to both a user query and the user's knowledge or interest. Not only does this form of IR further reduce information presented to the user based on what is most salient to them, but it can also lead a user towards information that is novel or of greater interest. Personalised query based summarisation systems also allow for more effective feedback loops then non personalised query based IR. The system can present with its summarisation its interpretation of the user interest or knowledge, facilitating interactive information retrieval \citep{chiaramella2000information}. The interpretation of knowledge can then be adjusted by the user to provide a summary that better serves the users' information needs.

Generalised text summarisation methods coverage of key content suffers from the complexity of forming a summary of short length from a large set of documents \citep{goldstein2000multi} et al., 2000). Generalised text summarisation methods also fail to reflect term importance and relations when applied to documents in a specific domain. Thus domain specific text summarisation methods attempt to provide better concept extraction, document representation, and summary formation from the use of domain models, such as ontologies or semantic based knowledge bases. Domain specific text summarisation has been applied to domains such as medicine \citep{sarker2013approach} and law \citep{galgani2012combining}, but these domain models are hand crafted or supervised by experts from those fields. Domain specific summarisation is limited based on the model’s comprehension of the domain. General models such as knowledge bases (supervised models created from semantic relationships) fail to comprehend domain specific term relationships and term significance. Thus common-sense knowledge bases are unable to replace expert crafted ontologies. Systems that provide domain specific personalised automatic text summarisation also use domain models in the creation of user interest or knowledge models \citep{ge2012ontology}, these systems are twice as reliant on domain models and their coherency. The majority of content online does not relate to domains with supplied or are easily generated ontologies.

For example, Wikipedia offers over 6 million articles in English. Due to the plethora of relevant and related documents for a given topic in a large domain, as well as the cyclic relations between documents, information overload and disorientation are common problems experienced by users researching a particular topic. World War II as an example contains 26,388 directly related pages. Even within a small domain such as the Watergate Scandal, there are 32 relevant pages. Some pages provide a general overview of a topic, but when reading or learning on a sub-topic of a broad topic, many of the pages that are related lay latent. These topics also lack formal ontologies so methods that reduce information overload from domain specific summarisation cannot be applied to them.

Therefore there is a need for domain specific text summarisation systems that are unsupervised. This would allow for the benefits of personalised domain specific summarisation to occur on the majority of content online, that do not have formalised domain models.

\section{Research Question}

The main problem with the application of domain specific personalised summarisation much of the domains of content on the internet is that existing methods are reliant on supervised domain models. While these existing methods have proven effective on domains with formal ontologies, the majority of available online content lacks formal ontologies. Work has been done on the automatic formation of ontologies \citep{bedini2007automatic}, but many of the state of the art techniques are reliant on curated domain corpuses, and others require expert validation. Automatically generated ontologies still fail to achieve the quality of domain expert generated ones. Therefore existing methods that work with expert generated ontologies, can only perform worse when used with automatically generated ones.

The tasks in automatically generating ontologies are similar to the formation of topic representations used in extractive text summarisation methods. Extractive summarisation methods create topic representation from: the extraction of terms, creation of topics based on semantic relationships or frequency of terms, and weighting relations of terms to topics. These tasks are very similar to those done in automatic ontology generation such as: extraction of concepts attributes and relations from a corpus source and analysis of extracted content to determine relations between content or ontologies, as described in \citep{bedini2007automatic}. These similarities suggest that existing extractive summarisation methods that use topic representations may lend themselves to performing domain specific personalised summarisation. This project explores this by addressing the following question:

\textbf{To what extent can existing automatic extractive summarization methods be used to provide domain specific personalised summaries, independent of domain specific ontologies and semantic models?}

The extent to which existing automatic extractive methods can be used is reliant on how well the proposed system can match the performance of state-of-the-art extractive summarisation methods. The extent is also reliant on systems ability to produce personalise summaries based on an given individual user knowledge model within a specific domain.

\section{Objectives}

The research question can be broken down into three objectives, which when completed, produced an unsupervised personalised domain independent summarisation system from existing systems and a determination of the efficacy of such a system. The three objectives are: to complete a review of the classifications, tasks, and methods of automatic summarisation; design and implement a system which performs domain specific personalised summaries without ontologies; and evaluate the extent to which the designed system is effective. These objectives together result in the answering of the research question. These three objectives are presented in detail below.

\subsection*{O1: Review of Automatic Text Summarisation}

The first objective is to perform a review of automatic text summarisation in order identify the classifications, methods, and approaches to tasks of summarisation. This review will be used in O2, to design a system that fulfills the requirements of the research question. To complete this objective system a broad review of the field of automatic text summarisation must be conducted. The aim of this review is to consider all approaches that may fulfill the requirements of the system. The review also serves to inform the approach of evaluating the proposed summarisation system. Thus the objectives of the literature view are:

\begin{itemize}
    \item Determine the classification of summarisation approaches.
    \item Identify the tasks involved in summarisation.
    \item Review of extractive summarisation methods.
    \item Review of evaluation methods for summarisation systems.
\end{itemize}

\subsection*{O2: Design and Implement a Summarization System}
To answer the research question a system must be designed and implemented which performs domain specific personalised summaries without ontologies. First the requirements, classifications and tasks need to be defined to outline a system which will achieve this objective. Existing methods of summarisation can then be identified from review of automatic text summarisation, to construct a summarisation system design from existing methods. The design must then be implemented, using the selected methods from their respective explanations. The objectives of the design and implementation objective for this project are:

\begin{itemize}
    \item the classification, and requirements needed by a system to perform domain specific personalised summarisation without formal ontologies
    \item Use existing methods of summarisation to identify a set of tasks that fit the classifications of the system and fulfill the requirements of the system.
    \item Select methods for the identified tasks from existing or a combination of existing approaches of automatic text summarisation to create a system design.
    \item From the design implement the selected methods, from their mathematical formalisation, algorithms or from libraries that contain the functionality, to produces a personalised summaries independent of domain models
\end{itemize}

\subsection*{O3: Evaluation of Proposed System}

The final objective of this project is to evaluate the extent to which the proposed system can perform personalised summaries in a given domain without the use of formal ontologies. This can be described by the following objectives:

\begin{itemize}
    \item Compare the performance of the proposed system to other state of the art systems in order to efficacy of the implementation and design.
    \item Examine the system's ability to personalise summaries based on specific model states to determine further work, and successful components of the system
\end{itemize}

\section{Overview}
This chapter has articulated the motivation, research question, and objectives of this project. The following chapters will build on this chapters foundation, further explaining the background, reasoning, and approach to determining the extent that existing extractive summarisation method can be used to perform specific domain personalised text summarisation without the need for supervised domain models. The system that was designed and implemented for this project, achieves competitive performance to other extractive systems and is found to be viable for use as a domain specific personalised summarisation system, independent of supervised domain models\footnote{Code for this project is found at: \url{https://github.com/jdcarbeck/FYP}}. The process to which the design of the proposed system was informed, developed, constructed and evaluated is presented in following chapters of this paper.

Chapter \ref{chp:2} presents the taxonomy, tasks, and approaches to automatic text summarisation. This chapter provides the necessary background for the reasoning used in the construction of the system design and for the methods used to implement the final system. The information reviewed in this chapter was used in informing the analysis and method for the design, implementation, and evaluation chapters.

Chapter 3 describes the methodology used in creating the design of the needed system. The classification and requirements of the system were defined inorder to identify the necessary tasks that the system needed to perform. Then methods from existing systems were selected to perform the identified tasks, based on the system requirements, compatibility with other methods, and performance. This chapter produces a system design that was used in the following implementation chapter.

Chapter 4 outlines the implementation of the proposed system from existing methods. The summarisation system was implemented from the process described in each method's respective literature. Some methods’ implementation was done via the use of libraries, while other methods required bespoke implementation based on the mathematical or pseudocode presented in the papers that present them. The implementation produced is a set of python classes which encapsulate selected methods for the system. The classes are used together to provide a system that performs personalised extractive summarisation  on domains without formal ontologies.

Chapter 5 presents the two methods used to determine the extent that this system can provide personalised summarisation on a domain independent of domain specific models. From the comparative evaluation of this system summarisation performance with state of the art extractive system, the system presents a competitive method of performing extractive summarisation. From the examination of the system on a specific domain set of document, the system is shown to produce interprobable personalised summaries from a context free and contextual queries, demonstrating that this system can be used to perform domain specific personalised summarisation on a specific domain.

Chapter 6 presents the limitation of the system and opportunities for future work for the design and evaluation.

Chapter 7 reviews the material presentend in this paper as well as the objectives set out for this project and the extent to which they were met. 

\RaggedRight