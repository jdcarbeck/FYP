\chapter{Conclusion}
\label{chp:7}
The goal of this project was to determine the extent to which existing methods of extractive summarisation could be used to construct a personalised domain specific summarisation system independent of formal ontologies. From a review of the automatic text summarisation a design was developed that used existing unsupervised extractive summarisation methods in order to design and then implement a system. This system was then evaluated to determine the extent of this system's performance of performing domain specific summarisation without the use of formal ontologies. Three objectives were set in order to achieve this goal. The objectives set from this project ensured that the system developed for this addressed the research question of this paper. The objectives from Chapter \ref{chp:1} and the conclusions on their completion are as follows:

\textbf{O1: Review of Automatic Text Summarisation}:
Inorder to inform the process used to design and evaluate a summarisation system that answers the research question, a review of the classification, tasks, and methods of automatic text summarisation need to be performed. Chapter \ref{chp:2} presented a review of automatic text summarisation. First the classifications of automatic text summarisation methods were presented. These classifications describe the input, purpose, method, and output of the summarisation system, and be used to ground the needed system within the field of automatic text summarisation. Next tasks that are universal to extractive summarisation systems were reviewed. The tasks presented are performed by all extractive systems. These tasks must be addressed by the system design inorder to perform extractive summarisation. Extractive methods of summarisation and their limitations were then presented, providing the set of methods to be considered for use in the final system design. Then approaches to performing query-based personalised summarisation were reviewed, presenting two main approaches to provide personalised summaries at a document and summary level. Lastly methods of evaluating automatic text summarisation systems were reviewed presenting the methods in which the design system could be evaluated. 

\textbf{O2: Design and Implement a Summarization System}:
Inorder to answer the research question of this project a system needed to be built that used existing methods of extractive summarisation to construct a system that performs domain specific personalised summarisation. In Chapter \ref{chp:3}, The review of the field of automatic text summarisation provided the tools in which a four step design process was used to develop a summarisation system design. First using the motivations of this project and the design objective specified from the research question, a set of requirements were defined. Using these requirements a set of classifications were defined based on the requirement. Second the tasks universal to extractive summarisation as well as tasks that help fulfill the requirements and classification of the system were identified, providing a high-level description of the systems design and functionality. Last methods were selected to perform the identified tasks based on their performance, fulfillment of requirements, and compatibility with other methods to be used in the system. In Chapter \ref{chp:4} the design was implemented from encapsulating the selected method into python classes. These classes were combined to perform the 3 main processes of the system, which perform personalised automatic text summarisation.

\textbf{O3: Evaluation of Proposed System}:
Inorder to determine the extent in which extractive methods can be combined to perform domain specific personalised summarisation, two evaluations were performed on the implemented system design. First a comparative evaluation was performed of the proposed system against existing state-of-the-art extractive summarisation systems. Summarisation datasets were used to calculate ROUGE scores of the proposed system summaries. These scores were compared against state-of-the-art extractive systems which were evaluated on the same datasets. The comparative analysis found that the proposed system had competitive performance for multi-document summarization. The system was then evaluated using both a context-free and contextual queries to examine the systems performance in producing personalised summaries. The system performed well on context free queries demonstrating its efficacy for providing generic personalised domain specific summaries. The system also demonstrated personalisation of summaries from contextual queries that had a common context. Though the system had some reduced performance in producing relevant readable summaries based on a shared context, the limitations examined were a result of the LDA topic representation and the small corpus used to evaluate the system. Thus it was concluded that the system formed from extractive methods of summarisation was effective in performing personalised domain specific summaries, indepent from domain ontologies.

From achieving these objectives a personalised domain specific summarisation system was designed, implemented, and then evaluated. The result of this process is a competitive extractive summarisation system that effectively performs personalised summarisation of specific domain material without the need of formal ontologies

